% !TeX root = ../main.tex

\section{图像生成模型的设计与算法实现}
\label{sec:implementation}

本章将详细阐述基于去噪扩散概率模型(DDPM)的生成系统设计。我们将从网络架构的微观设计入手,结合 `ddpm\_torch` 代码库的实际实现,探讨如何构建能够有效预测噪声的神经网络,进而介绍条件控制机制的扩展方式,最后通过形式化的伪代码描述模型的训练与推理流程,并给出具体的实验参数配置。
\subsection{模型整体架构设计}

扩散模型的核心任务是训练一个噪声预测网络 $\boldsymbol{\epsilon}_{\theta}(\mathbf{x}_t, t)$。由于该任务需要输入和输出具备相同的空间分辨率,且需在不同尺度上捕捉图像特征,本研究采用了基于 U-Net \cite{ronneberger2015u} 的改进架构。该架构通过“编码器-解码器”结构实现特征的多尺度提取与融合,并结合了现代卷积神经网络的设计原则,如残差连接、组归一化(Group Normalization)和自注意力机制。

\subsubsection{噪声预测网络总体结构}
根据代码实现细节,我们的网络架构主要由以下五个部分组成:

\begin{enumerate}
	\item **特征投影层(Input Projection)**:输入图像 $\mathbf{x}_t$ 首先通过一个 $3 \times 3$ 的卷积层(Padding=1)映射到初始通道维度(如 $C=128$),不仅保留了空间维度,还完成了初步的特征升维。
	
	\item **下采样编码器(Downsampling Encoder)**:由多个分辨率层级(Levels)组成。每个层级包含 $N$ 个残差模块(Residual Blocks),部分层级后接下采样模块。随着层级加深,特征图的空间分辨率逐步减半,通道数则按预设倍率(如 $1:2:2:2$)增加。
	
	\item **中间瓶颈层(Middle Block)**:位于 U-Net 的最底层,用于处理最低分辨率的特征图。该模块采用了“残差块 $\rightarrow$ 自注意力块 $\rightarrow$ 残差块”的堆叠结构,旨在捕捉图像的全局上下文信息。
	
	\item **上采样解码器(Upsampling Decoder)**:结构与编码器对称。每个层级首先将上一层的特征图进行上采样,并与编码器中对应层级的特征图进行通道拼接(Concatenation),随后通过 $N+1$ 个残差模块进行特征融合与重建。
	
	\item **输出预测层(Output Head)**:经过解码器恢复至原始分辨率的特征图,依次通过组归一化、SiLU 激活函数和一个 $3 \times 3$ 卷积层,最终输出与输入图像同形状的噪声预测值 $\boldsymbol{\epsilon}_{\text{pred}}$。
\end{enumerate}

\subsubsection{基于 U-Net 的特征提取与上采样模块设计}
为了提升模型的生成质量和训练稳定性,我们在基础模块的设计上进行了以下针对性优化:

**1. 预激活残差模块(Pre-activation Residual Block)**\\
不同于传统的 ResNet 结构,本研究采用了宽残差网络(Wide ResNet)的变体。每个残差块包含两个卷积层,并在卷积操作前先行进行归一化和激活(Norm $\rightarrow$ Act $\rightarrow$ Conv)。具体配置如下:
\begin{itemize}
	\item **归一化**:统一采用 Group Normalization (GN),设置组数 $G=32$,$\epsilon=10^{-6}$。相比于 Batch Normalization,GN 对 Batch Size 的变化不敏感,更适合生成任务。
	\item **激活函数**:采用 SiLU (Sigmoid Linear Unit, $f(x)=x \cdot \sigma(x)$) \cite{elfwing2018sigmoid},其平滑非单调的特性有助于梯度的深层传播。
	\item **时间步注入**:时间步嵌入向量 $\mathbf{e}_t$ 经由一个线性层投影后,以**逐元素相加(Element-wise Sum)**的方式注入到第一个卷积层的输出特征中,从而调制网络对不同噪声水平的响应。
	\item **Dropout**:在两个卷积层之间引入 Dropout(比率 $p=0.1$)以防止过拟合。
\end{itemize}

**2. 采样策略优化**\\
为了减少棋盘格效应(Checkerboard Artifacts),我们摒弃了传统的最大池化和转置卷积:
\begin{itemize}
	\item **下采样**:采用步长为 2 的 $3 \times 3$ 卷积层(配合特定 Padding 保持特征对齐)。
	\item **上采样**:采用最近邻插值(Nearest Neighbor Interpolation)将特征图放大 2 倍,随后接一个 $3 \times 3$ 卷积层以平滑特征。
\end{itemize}

**3. 自注意力机制(Self-Attention)**\\
在低分辨率层级(如 $16 \times 16$),我们引入了多头自注意力模块。特征图 $\mathbf{h}$ 首先通过 $1 \times 1$ 卷积映射为查询 $Q$、键 $K$ 和值 $V$。注意力权重由 $A = \text{softmax}(\frac{QK^T}{\sqrt{d_k}})$ 计算,最终输出为 $\mathbf{h} + \text{Proj}_{\text{out}}(AV)$。为了优化初始训练阶段的梯度流,输出投影层 $\text{Proj}_{\text{out}}$ 的权重被初始化为零。

\subsubsection{时间步嵌入的位置编码实现}
由于 $p_\theta(\mathbf{x}_{t-1}|\mathbf{x}_t)$ 的分布特性随 $t$ 剧烈变化,网络必须精确感知当前的时间步。我们采用 Transformer 中的正弦位置编码(Sinusoidal Positional Encoding)将标量 $t$ 映射为高维向量 $\mathbf{e}_t$。

具体而言,首先将时间步 $t$ 编码为固定维度的正弦向量:
\begin{equation}
	\mathbf{e}_{\text{raw}}^{(2i)} = \sin\left(\frac{t}{10000^{2i/d}}\right), \quad \mathbf{e}_{\text{raw}}^{(2i+1)} = \cos\left(\frac{t}{10000^{2i/d}}\right)
\end{equation}
其中 $d$ 为基础通道数的 4 倍。随后,该向量通过一个由两层全连接层(Linear)和 SiLU 激活函数组成的多层感知机(MLP)进行变换,得到最终的时间步嵌入 $\mathbf{e}_t = \text{MLP}(\mathbf{e}_{\text{raw}})$。这一设计使得网络能够学习到时间步之间复杂的非线性依赖关系。

\subsection{条件控制机制的实现}

为了实现可控图像生成(如生成指定类别的数字或物体),我们在基础 U-Net 架构上扩展了条件控制接口。本研究采用了**类别嵌入(Class Embedding)**与**无分类器引导(Classifier-Free Guidance, CFG)**相结合的策略,这种设计无需在训练完成后额外引入独立的分类器,即可显著提升生成样本与目标类别的一致性。

\subsubsection{类别信息的向量化与融合}
在条件扩散模型中,类别标签 $y \in \{1, \dots, K\}$ 不再是简单的标量索引,而是作为一种全局上下文信号注入网络。具体的嵌入与融合流程如下:

\begin{enumerate}
	\item **类别嵌入映射**:首先,通过一个可学习的嵌入层(Embedding Layer)将离散的类别标签 $y$ 映射为致密的特征向量 $\mathbf{v}_y \in \mathbb{R}^{d_{\text{emb}}}$。为了保证特征融合的兼容性,该向量的维度 $d_{\text{emb}}$ 被设计为与时间步嵌入 $\mathbf{e}_t$ 的维度一致(即基础通道数的 4 倍)。
	
	\item **空标签处理**:为了支持无分类器引导算法,我们在嵌入层中预留了一个特殊的“空标签”(Null Token)$\emptyset$(通常索引设为 $0$ 或 $K+1$)。该标签对应的嵌入向量 $\mathbf{v}_\emptyset$ 同样是可学习的参数,用于表示无条件生成时的状态。
	
	\item **时序-类别特征融合**:在输入到 U-Net 的各个残差模块之前,我们将类别向量与时间步向量进行融合。本研究采用**逐元素相加(Element-wise Sum)**作为融合策略,即最终的上下文嵌入向量 $\mathbf{e}_{\text{ctx}}$ 计算如下:
	\begin{equation}
		\mathbf{e}_{\text{ctx}} = \text{MLP}_t(\mathbf{e}_t) + \text{MLP}_y(\mathbf{v}_y)
	\end{equation}
	其中 $\text{MLP}_t$ 和 $\text{MLP}_y$ 分别是针对时间步和类别的投影网络。这种加法融合策略利用了高维空间中不同语义特征的正交性,使得网络能够同时根据当前的噪声水平 $t$ 和目标类别 $y$ 动态调整特征图的分布,而无需显著增加计算开销。
\end{enumerate}

\subsubsection{无分类器引导(Classifier-Free Guidance)的算法实现}
传统的条件生成依赖于一个在噪声图像上训练的分类器 $p_\phi(y|\mathbf{x}_t)$ 来提供梯度指引,这不仅增加了训练成本,还容易引入对抗攻击般的梯度假象。本研究采用无分类器引导(CFG)技术,其核心思想是利用生成模型本身来隐式地构建分类器梯度。

**1. 联合训练策略**\\
在训练阶段,我们并不总是向网络提供真实的类别标签 $y$。而是以一个固定的概率 $p_{\text{uncond}}$(本实验设为 0.1)将标签 $y$ 随机替换为空标签 $\emptyset$。这使得同一个神经网络 $\boldsymbol{\epsilon}_\theta$ 能够同时学习两个分布:
\begin{itemize}
	\item **条件分布** $\boldsymbol{\epsilon}_\theta(\mathbf{x}_t, t, y)$:当输入真实标签时学习。
	\item **无条件分布** $\boldsymbol{\epsilon}_\theta(\mathbf{x}_t, t, \emptyset)$:当输入空标签时学习,近似于数据本身的边际分布。
\end{itemize}

**2. 采样阶段的梯度外推**\\
根据贝叶斯准则,隐式分类器的梯度 $\nabla_{\mathbf{x}_t} \log p(y|\mathbf{x}_t)$ 可以表示为条件得分与无条件得分之差。在推理(采样)阶段,我们将预测噪声修正为上述两者的线性组合:
\begin{equation}
	\tilde{\boldsymbol{\epsilon}}_\theta(\mathbf{x}_t, t, y) = \boldsymbol{\epsilon}_\theta(\mathbf{x}_t, t, \emptyset) + w \cdot \underbrace{\left( \boldsymbol{\epsilon}_\theta(\mathbf{x}_t, t, y) - \boldsymbol{\epsilon}_\theta(\mathbf{x}_t, t, \emptyset) \right)}_{\text{隐式分类器梯度方向}}
	\label{eq:cfg_formula}
\end{equation}
其中 $w$ 为引导尺度(Guidance Scale)。
\begin{itemize}
	\item 当 $w=1$ 时,退化为标准的条件生成。
	\item 当 $w > 1$ 时,模型沿梯度方向进行“外推”,强化了条件信号 $y$ 的影响,使得生成的图像特征更典型,但可能会降低样本的多样性。
\end{itemize}

这种实现方式巧妙地规避了专门训练噪声分类器的需求,仅需在推理时进行两次前向传播(一次有条件,一次无条件),即可获得高质量的条件生成结果。

\subsection{模型超参数配置与实现细节}

为了确保实验的可复现性,本节列出了模型在 CIFAR-10 和 CelebA 数据集上的具体参数配置。所有超参数的设定均参考了 `configs/` 目录下的标准配置文件,并根据硬件环境进行了适配。

\subsubsection{网络结构参数}
网络架构基于改进的 U-Net。表 \ref{tab:model_params} 以 CIFAR-10 数据集为例,详细列出了关键的结构参数。值得注意的是,对于 CelebA 数据集($64 \times 64$),虽然基础通道数保持为 128,但为了防止过拟合,我们将 Dropout 概率调整为 0.0,且注意力层依旧施加在 $16 \times 16$ 分辨率的特征图上。

\begin{table}[htbp]
	\centering
	\caption{模型网络结构参数配置 (以 CIFAR-10 为例)}
	\label{tab:model_params}
	\begin{tabular}{l|c}
		\hline
		\textbf{参数名称} & \textbf{配置值} \\
		\hline
		基础通道数 (Base Channels) & 128 \\
		通道倍率 (Channel Multipliers) & $[1, 2, 2, 2]$ \\
		残差块数量 (ResBlocks per Scale) & 2 \\
		注意力应用层级 (Apply Attention) & $16 \times 16$ 分辨率层 \\
		时间嵌入维度 (Time Embed Dim) & $128 \times 4 = 512$ \\
		Dropout 概率 & 0.1 (CelebA 为 0.0) \\
		归一化方式 & GroupNorm (32 groups) \\
		激活函数 & SiLU \\
		\hline
	\end{tabular}
\end{table}

\subsubsection{扩散过程与训练参数}
前向扩散过程采用线性方差调度策略(Linear Schedule),并结合指数移动平均(EMA)技术以稳定生成质量。具体设置如下:

\begin{itemize}
	\item **时间步设置**:总时间步数 $T = 1000$。
	\item **方差策略**:方差范围从 $\beta_1 = 1 \times 10^{-4}$ 线性增加至 $\beta_T = 0.02$。在逆向过程中,CIFAR-10 模型采用 `fixed-large` 方差策略(即 $\sigma_t^2 = \beta_t$),而 CelebA 模型采用 `fixed-small` 策略(即 $\sigma_t^2 = \tilde{\beta}_t$),以适应不同数据集的分布特性。
	\item **优化器**:使用 Adam 优化器,参数设定为 $\beta_1=0.9, \beta_2=0.999$,初始学习率设为 $2 \times 10^{-4}$。
	\item **学习率调度**:训练初期设置 5000 步的线性预热(Warmup),随后保持恒定。
	\item **梯度裁剪**:设置梯度范数阈值(Grad Norm Clip)为 1.0,防止梯度爆炸。
	\item **EMA 策略**:训练过程中维护一个影子模型,其参数衰减率(Decay)设为 0.9999,推理时使用 EMA 参数。
\end{itemize}

在硬件环境方面,模型训练在单张 **NVIDIA GeForce RTX 5090 (32GB VRAM)** GPU 上进行。得益于 RTX 5090 强大的显存容量与计算能力,我们能够使用较大的批量大小(Batch Size = 128)进行训练,并结合混合精度(Mixed Precision)技术,在保证数值稳定性的同时显著缩短了收敛时间。