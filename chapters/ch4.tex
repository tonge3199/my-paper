% chapter4.tex
\chapter{实验结果与分析}
\label{chap:experiment}

\section{实验环境与设置}

  \subsection{软硬件环境介绍}
  % PyTorch version, GPU model (e.g., RTX 3060/Colab Tesla T4)

  \subsection{数据集选取与预处理}
  % MNIST / CIFAR-10 description

  \subsection{评价指标说明}
  % FID, IS, Loss Curve explanation

\section{模型训练过程分析}

  \subsection{损失函数收敛曲线分析}
	% Insert Loss Curve Image
	% \begin{figure}[htbp]
		%     \centering
		%     \includegraphics[width=0.8\textwidth]{figures/loss_curve.png}
		%     \caption{模型训练损失函数收敛曲线}
		%     \label{fig:loss_curve}
		% \end{figure}

  \subsection{不同训练阶段的中间结果可视化}
  % Show noisy images at different epochs

\section{图像生成质量验证}

  \subsection{无条件生成结果展示与质量评估}
  % Show grid of generated images

  \subsection{条件控制生成效果演示}
  % Show specific class generation (e.g., generating only "cats" or number "5")

\section{对比实验与消融研究}

  \subsection{不同采样步数对生成质量与速度的影响}
  % Compare T=1000 vs T=50 (DDIM)

  \subsection{不同网络深度/宽度对模型性能的影响分析}
  % Optional: Comparison table