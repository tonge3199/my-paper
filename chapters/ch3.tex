% chapter3.tex
\chapter{图像生成模型的设计与算法实现}
\label{chap:implementation}

\section{模型整体架构设计}

  \subsection{噪声预测网络总体结构}
  % Describe the overall architecture

  \subsection{基于U-Net的特征提取与还原模块设计}
  % Detailed U-Net design, Downsample, Upsample blocks
  本研究采用改进的 U-Net 结构作为噪声预测网络 $\epsilon_\theta$……

  \subsection{时间步嵌入(Time Embedding)的位置编码实现}
  % Sinusoidal positional embeddings
  为了使网络感知当前的扩散步数 $t$,采用了正弦位置编码……

\section{条件控制机制的实现}

  \subsection{类别信息的嵌入方法}
  % Class Embedding

  \subsection{无分类器引导(Classifier-Free Guidance)的实现原理}
  % CFG implementation details

\section{核心算法的伪代码与程序流程}

  \subsection{训练阶段:随机时间步采样与梯度下降算法}
  % You can use the algorithm environment here
	\begin{algorithm}
		\caption{DDPM 训练过程}
		\begin{algorithmic}
			\STATE \textbf{Input:} Training data distribution $q(x_0)$
			\STATE \textbf{Repeat:}
			\STATE \quad Sample $x_0 \sim q(x_0)$
			\STATE \quad Sample $t \sim \text{Uniform}(\{1, \dots, T\})$
			\STATE \quad Sample $\epsilon \sim \mathcal{N}(0, \mathbf{I})$
			\STATE \quad Take gradient descent step on $\nabla_\theta \| \epsilon - \epsilon_\theta(\sqrt{\bar{\alpha}_t} x_0 + \sqrt{1-\bar{\alpha}_t}\epsilon, t) \|^2$
			\STATE \textbf{Until} converged
		\end{algorithmic}
	\end{algorithm}

  \subsection{推理阶段:逐步去噪与图像生成的迭代算法}